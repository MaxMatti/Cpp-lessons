\documentclass[12pt, a4paper]{article}
\usepackage{a4wide}
\usepackage{DejaVuSans}
\usepackage{listings}
\usepackage{amssymb}
\usepackage{amsmath}
\renewcommand*\familydefault{\sfdefault} 
\lstset{basicstyle=\ttfamily}
\begin{document}
\section*{BigInteger-Library}
\begin{itemize}
    \item Create a class called \lstinline{BigInteger} that should represent large integer values with the following features
    \item Your class should be able to store integer values of any compile-time defined size (runtime-defined size is possible but not required)
    \item When templated, your class should either be able to deduce its template parameters or you should provide a derived template that only takes the size (in bits) as a template parameter
    \item Any part of your program should be platform-independent C++-Code according to the C++14 or C++17 specification
\end{itemize}
\section*{First Stage}
\begin{itemize}
    \item Implement the following operations for your class:
    \begin{itemize}
        \item (In)Equality-check
        \item Comparison (less-than, etc)
        \item Addition, Subtraction
        \item Bit-shift
        \item Constructor that takes a 64 bit sized integer value as input
    \end{itemize}
\end{itemize}
\section*{Second Stage}
\begin{itemize}
    \item Implement the following operations for your class:
    \begin{itemize}
        \item Multiplication
        \item Division
        \item Modulo (remainder of division)
    \end{itemize}
\end{itemize}
\section*{Third Stage}
\begin{itemize}
    \item Implement a square-and-multiply-based algorithm to calculate nth power in a residue class ring
    \item Make sure your \lstinline{BigInteger} class is printable (can be appended to an \lstinline{std::ostream})
    \item Create a constructor that takes a \lstinline{std::string} as input
    \item Make sure your \lstinline{BigInteger} class can be constructed from a \lstinline{std::istream}
\end{itemize}
\section*{Evaluation Criteria}
\begin{itemize}
    \item Functionality:
    \begin{itemize}
        \item The code needs to compile and run without errors in C++14 or C++17
        \item The code needs to provide the required functionality and follow the specifications
    \end{itemize}
    \item Readability:
    \begin{itemize}
        \item The implemented functions need to be comprehensible
        \item The structure of the source code should be obvious and reasonable
    \end{itemize}
    \item Best pratices:
    \begin{itemize}
        \item Use known best practices for the placement and hierarchy of implemented functions
        \item No functionality should be implemented twice or even more often
        \item Avoid compiler warnings where possible, don't create quick workarounds for compiler warnings
    \end{itemize}
    \item Performance:
    \begin{itemize}
        \item Faster and more efficient code is preferred
        \item This should not come with a tradeoff of readability
        \item Real time performance as well as theoretical complexity is measured
    \end{itemize}
\end{itemize}
\end{document}
