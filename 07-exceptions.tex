\documentclass[10pt, compress, usenames, dvipsnames]{beamer}

\usetheme{m}
\usepackage{booktabs}
\usepackage[scale=2]{ccicons}
\usepackage{minted}
\usepackage{listings}
\usepackage{scrextend}

% set the default code style
\lstset{
    frame=tb, % draw a frame at the top and bottom of the code block
    tabsize=4, % tab space width
    showstringspaces=false, % don't mark spaces in strings
    numbers=left, % display line numbers on the left
    commentstyle=\color{OliveGreen}, % comment color
    keywordstyle=\color{MidnightBlue}, % keyword color
    stringstyle=\color{WildStrawberry}, % string color
	language=C++,
	basicstyle=\ttfamily
}

\usemintedstyle{trac}

\usepackage{colortbl}

\begin{document}

\title{Exceptions}
\subtitle{How to deal with errors}
\date{\today}
\author{Florian Warg, Max Staff}

\maketitle

\begin{frame}[fragile]
    \frametitle{Exception handling}
    \begin{itemize}
    \item exception handling is a way of dealing with runtime errors
    \item if an exception occurs, control is transferred to handlers
    \item exceptions may be handled gracefully so program execution can continue
    \item exceptions can occur:
    \begin{itemize}
        \item when trying to allocate memory
        \item when providing an illegal argument to a function
        \item when we break math (divide by zero)
        \item when we dynamically cast to unrelated types
        \item when a variable overflows or underflows
        \item ...
    \end{itemize}
    \end{itemize}
\end{frame}

\begin{frame}[fragile]
    \frametitle{Try and Catch}
    \begin{itemize}
        \item critical parts of program must be surrounded by \lstinline{try} block
        \item this is followed by a \lstinline{catch} block that handles the exception
    \end{itemize}
    \begin{lstlisting}[numbers=none]
void evil() { throw exception(); }
/* ... */
try {
    evil();
} catch (const exception& e) {
    cerr << e.what() << "\n";
}
    \end{lstlisting}
\end{frame}

\begin{frame}[fragile]
    \frametitle{Pre-defined exceptions}
    \begin{itemize}
        \item \lstinline{logic_error (invalid_argument, ...)}
        \item \lstinline{runtime_error (overflow_error, ...)}
        \item \lstinline{bad_typeid}
        \item \lstinline{bad_cast}
        \item \lstinline{bad_weak_ptr}
        \item \lstinline{bad_function_call}
        \item \lstinline{bad_alloc}
        \item \lstinline{bad_exception}
        \item \lstinline{ios_base::failure}
        \item ...
    \end{itemize}
    \href{All exceptions on cppreference}{http://en.cppreference.com/w/cpp/error/exception}
\end{frame}

\begin{frame}[fragile]
    \frametitle{Create your own exceptions}
    \begin{itemize}
    \item you can derive from \lstinline{std::exception}
    \item your class has to implement the \lstinline{what()} function
    \begin{lstlisting}[numbers=none]
class MyException : public std::exception {
    virtual const char* what() const override {
        return "My exception was thrown!\n";
    }
};
    \end{lstlisting}
    \end{itemize}
\end{frame}

\begin{frame}[fragile]
    \frametitle{Handling different exceptions}
    \begin{itemize}
    \item you can use several \lstinline{catch} blocks to handle different exceptions
    \begin{lstlisting}[numbers=none]
void evil(int x) {
    if (x < 0)
        throw std::exception();
    else
        throw MyException();
}
/* ... */
try { evil(1); } 
catch (std::exception e) { /* ... */ } 
catch (MyException e) { /* ... */ }
    \end{lstlisting}
    \end{itemize}
\end{frame}

\begin{frame}[fragile]
    \frametitle{Throwing anything}
    \begin{itemize}
    \item you are not limited to exception objects
    \item you can actually throw \textbf{anything}
    \item (but why would you want to do that)
    \begin{lstlisting}[numbers=none]
void evil() { throw 42; }
/* ... */
try { evil(); } 
catch (int i) {
    cerr << i << '\n';
}
    \end{lstlisting}
    \end{itemize}
\end{frame}

\begin{frame}[fragile]
    \frametitle{Catching anything}
    \begin{itemize}
    \item because of polymorphism you can catch all objects of a class hierarchy by reference
    \item you can also catch anything that is thrown
    \begin{lstlisting}[numbers=none]
void evil() { throw MyException(); }
/* ... */
try { evil(); 
} catch (std::exception& e) {
    /* also catches MyException */
} catch (...) {
    /* catches anything */
}
    \end{lstlisting}
    \end{itemize}
\end{frame}

\begin{frame}[fragile]
    \frametitle{Exception safety guarantees}
    \begin{itemize}
        \item there are 4 levels of exception guarantees in C++
        \item the higher safety guarantees make it easy to recover from exceptions
        \item levels are in decreasing order (level 1 is the highest safety guarantee)
    \end{itemize}
\end{frame}

\begin{frame}[fragile]
    \frametitle{Level 1: No-throw guarantee}
    \begin{itemize}
        \item function does not throw exceptions even in exceptional situations
        \item occuring exceptions are handled internally
        \item function will success in every situation
        \item keyword \lstinline{noexcept} can be used to mark functions
    \end{itemize}
    \begin{lstlisting}[numbers=none]
int f() noexcept { return 42; }
    \end{lstlisting}
\end{frame}


\begin{frame}[fragile]
    \frametitle{Level 2: Strong exception safety}
    \begin{itemize}
        \item also known as commit or rollback semantics
        \item function can fail but is guaranteed to have no side effects
        \item if this function fails, all data will retain their original values
    \end{itemize}
\end{frame}

\begin{frame}[fragile]
    \frametitle{Level 3: Basic exception safety}
    \begin{itemize}
        \item also known as no-leak guarantee
        \item failed function can have side effects but invariants are preserved and resources are not leaked
    \end{itemize}
\end{frame}

\begin{frame}[fragile]
    \frametitle{Level 4: No exception safety}
    \begin{itemize}
        \item no guarantees are made \texttrademark
    \end{itemize}
\end{frame}

\end{document}
