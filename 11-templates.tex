\documentclass[10pt, compress, usenames, dvipsnames]{beamer}

\usetheme{m}
\usepackage{booktabs}
\usepackage[scale=2]{ccicons}
\usepackage{minted}
\usepackage{listings}
\usepackage{scrextend}

% set the default code style
\lstset{
    frame=tb, % draw a frame at the top and bottom of the code block
    tabsize=4, % tab space width
    showstringspaces=false, % don't mark spaces in strings
    numbers=left, % display line numbers on the left
    commentstyle=\color{OliveGreen}, % comment color
    keywordstyle=\color{MidnightBlue}, % keyword color
    stringstyle=\color{WildStrawberry}, % string color
	language=C++,
	basicstyle=\ttfamily
}

\usemintedstyle{trac}


\begin{document}

\title{Templates}
\subtitle{Generic Programming}
\date{\today}
\author{Florian Warg, Max Staff}

\maketitle

\begin{frame}[fragile]
    \frametitle{Motivation}
    \begin{itemize}
        \item often you discover that your algorithm is generic
        \item you want to apply it to any type that conforms to a concept
    \end{itemize}
    \begin{lstlisting}[numbers=none]
int min(int x, int y)
{
    return (x < y) ? x : y;
}
    \end{lstlisting}
\end{frame}

\begin{frame}[fragile]
    \frametitle{Approach 1: Function Overloading}
    \begin{lstlisting}[numbers=none]
int min(int x, int y) { return (x < y) ? x : y; }
char min(char x, char y) { return (x < y) ? x : y; }
    \end{lstlisting}
    \begin{itemize}
        \item these are all the same
        \item there must be an overload for every type
    \end{itemize}
\end{frame}

\begin{frame}[fragile]
    \frametitle{Approach 2: Templates}
    \begin{itemize}
		\item function should be defined for every comparable type
		\item the powerful template engine takes care of that
    \end{itemize}
    \begin{lstlisting}[numbers=none]
template<class T>
T min(T x, T y) { return (x < y) ? x : y; }
/* ... */
auto i = min(42, 1);
auto k = min(34.0, 23.1);
auto l = min(make_unique<int>(2), nullptr);
    \end{lstlisting}
\end{frame}

\begin{frame}[fragile]
    \frametitle{Template syntax}
    \begin{itemize}
         \item template parameters can be values, but also types
         \item templates can be applied to types and functions
    \end{itemize}
    \begin{lstlisting}[numbers=none]
template<class T>
T min(T x, T y);

template<class T, std::size_t N>
struct array { T data[N]; };
    \end{lstlisting}
\end{frame}

\begin{frame}[fragile]
    \frametitle{Type deduction}
    \begin{itemize}
        \item template parameters can / must be explicitly stated
        \item in C++14 function template parameters can also be deduced
    \end{itemize}
    \begin{lstlisting}[numbers=none]
auto i = min<int>(41, 1);
auto k = min<double>(34.0, 23.1);
auto l = min<double>(10, 0.0); // parameter needed!

auto x = min(23.0f, 23.01f);
array<int, 8> a; // parameters needed!
    \end{lstlisting}
\end{frame}

\begin{frame}[fragile]
    \frametitle{Overload resolution}
    \begin{itemize}
        \item the compiler prefers concrete functions to generic ones
        \item template functions can be explicitly called
    \end{itemize}
    \begin{lstlisting}[numbers=none]
void fn(int x) { cout << "concrete\n"; }

template<class T>
void fn(T x) { cout << "templated\n"; }

fn(42); // "concrete"
fn(10.0); // "templated"
fn<int>(42); // "templated"
    \end{lstlisting}
\end{frame}

\begin{frame}[fragile]
    \frametitle{Template Specialization}
    \begin{itemize}
        \item templates can be specialized
        \item you provide a special implementation for specific parameters
    \end{itemize}
    \begin{lstlisting}[numbers=none]
template<class T>
bool isInt(T t) { return false; }

template<>
bool isInt<int>(int t) { return true; }
    \end{lstlisting}
\end{frame}

\begin{frame}[fragile]
    \frametitle{Template instantiation}
    \begin{itemize}
        \item templates can only be instantiated when their expressions are well formed
        \item if a template cannot be instantiated, alternative templates will be considered
        \item this is called \textbf{Substitution Failure Is Not An Error} (SFINAE)
        \item if no template can be instantiated, a compiler error occurs
    \end{itemize}
    \begin{lstlisting}[numbers=none]
template<class T>
void call(T t) { t(); }

void f() { cout << "f called!"; }

call(f);
call([](){ cout << "lambda called!"; });
call(42); // 42() is ill formed! (Substitution failure)

    \end{lstlisting}
\end{frame}

\end{document}
