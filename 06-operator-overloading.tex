\documentclass[10pt, compress, usenames, dvipsnames]{beamer}

\usetheme{m}
\usepackage{booktabs}
\usepackage[scale=2]{ccicons}
\usepackage{minted}
\usepackage{listings}
\usepackage{scrextend}

% set the default code style
\lstset{
    frame=tb, % draw a frame at the top and bottom of the code block
    tabsize=4, % tab space width
    showstringspaces=false, % don't mark spaces in strings
    numbers=left, % display line numbers on the left
    commentstyle=\color{OliveGreen}, % comment color
    keywordstyle=\color{MidnightBlue}, % keyword color
    stringstyle=\color{WildStrawberry}, % string color
	language=C++,
	basicstyle=\ttfamily
}

\usemintedstyle{trac}

\usepackage{colortbl}

\begin{document}

\title{Operator overloading}
\subtitle{Using standard library functions and providing an intuitive interface}
\date{\today}
\author{Florian Warg, Max Staff}

\maketitle

\begin{frame}[fragile]
    \frametitle{How do you sort your own objects?}
    \begin{itemize}
    \item do not use your own containers
    \item do not build your own sorting function
    \item use the standard library
    \visible<2->{\item solution: build own comparison operators}
    \visible<3->{\item also helps for making your object "printable"...}
    \end{itemize}
\end{frame}

\begin{frame}[fragile]
    \frametitle{Declaring an operator I}
    \begin{lstlisting}[numbers=none]
class Gear {
    float gearRatio;
public:
    float getRatio() { return this->gearRatio; }
    bool operator<(const Gear& other);
    bool operator==(const Gear& other);

    bool operator<=(const Gear& other);
    bool operator>=(const Gear& other);
    bool operator>(const Gear& other);
    bool operator!=(const Gear& other);
}
    \end{lstlisting}
\end{frame}

\begin{frame}[fragile]
    \frametitle{Declaring an operator II}
    \begin{lstlisting}[numbers=none]
class Matrix {
    int content[width * height];
public:
    int* operator[](std::size_t index);
    Matrix operator+(const Matrix& other) const;
    Matrix& operator+=(const Matrix& other);
}

Matrix a, b;
a += b;
Matrix c = a + b;
    \end{lstlisting}
\end{frame}

\begin{frame}[fragile]
    \frametitle{Declaring an operator III}
    \begin{lstlisting}[numbers=none]
class Matrix {
    int content[width * height];
public:
    Matrix& operator++();
    Matrix operator++(int);
    Matrix& operator--();
    Matrix operator--(int);
}

Matrix a;
++a;
a++;
--a;
a--;
    \end{lstlisting}
\end{frame}

\begin{frame}[fragile]
    \frametitle{Overloading an increment operator}
    \begin{lstlisting}[numbers=none]
Matrix& Matrix::operator++() {
    for (auto& i : this->content) {
        ++i; // increment
    }
    return *this; // return incremented state
}

Matrix Matrix::operator++(int) {
    Matrix tmp(*this); // copy
    this->operator++(); // increment
    return tmp; // return previous state
}

Matrix a;
++a;
a++;
    \end{lstlisting}
\end{frame}

\begin{frame}[fragile]
    \frametitle{Overloading a function operator}
    \begin{lstlisting}[numbers=none]
class Matrix {
    int content[width * height];
public:
    Matrix& operator()(std::size_t x, std::size_t y);
}

int Matrix::operator()(std::size_t x, std::size_t y) {
    return this->content[y * width + x];
}

Matrix a;
int b = a(5, 3);
    \end{lstlisting}
\end{frame}

\begin{frame}[fragile]
    \frametitle{Functors}
    \begin{itemize}
    \item Functor = object with overloaded \lstinline{()}-operator
    \item better optimizable than function-pointers
    \item useful with templates (coming soon)
    \end{itemize}
\end{frame}

\begin{frame}[fragile]
    \frametitle{Which operators are overloadable?}
    \begin{lstlisting}[numbers=none]
+       -       *       /       %
+=      -=      *=      /=      %=

&       |       ^       ~
&=      |=      ^=      

<<      >>      <       >       ==
<<=     >>=     <=      >=      !=

!       &&      ||      ++      --
->*     ->      ,       ()      []
    \end{lstlisting}
\end{frame}

\begin{frame}[fragile]
    \frametitle{Which operators are not overloadable?}
    \begin{itemize}
    \item \lstinline{::} scope resolution
    \item \lstinline{.} member access
    \item \lstinline{.*} member access through pointer to member
    \item \lstinline{?:} ternary conditional
    \visible<2->{\item new or nonexistent operators (\lstinline{**, <>, &|})}
    \visible<3->{\item \lstinline{&&, ||} lose short-circuit (aka "lazy") evaluation}
    \visible<4->{\item precedence, grouping or amount of operands cannot be changed}
    \end{itemize}
\end{frame}

\begin{frame}[fragile]
    \frametitle{Overloading outside of the class}
    \begin{tabular}{lll}
    \rowcolor{gray!50}
    expression & member function & non-member\\
    {\lstinline!@a!} & {\lstinline!(a).operator@()!} & {\lstinline!operator@(a)!} \\
    \rowcolor{gray!25}
    {\lstinline!a@b!} & {\lstinline!(a).operator@(b)!} & {\lstinline!operator@(a, b)!} \\
    {\lstinline!a=b!} & {\lstinline!(a).operator=(b)!} & \cellcolor{red!15}not possible \\
    \rowcolor{gray!25}
    {\lstinline!a(b...)!} & {\lstinline!(a).operator()(b...)!} & \cellcolor{red!25}not possible \\
    {\lstinline!a[b]!} & {\lstinline!(a).operator[](b)!} & \cellcolor{red!15}not possible \\
    \rowcolor{gray!25}
    {\lstinline!a->!} & {\lstinline!a.operator->()!} & \cellcolor{red!25}not possible \\
    {\lstinline!a@!} & {\lstinline!a.operator@(0)!} & {\lstinline!operator@(a, 0)!} \\
    \end{tabular}
\end{frame}

\begin{frame}[fragile]
    \frametitle{Example}
    \begin{lstlisting}[numbers=none]
class Matrix {
    int content[width * height];
    // ...
public:
    void print(std::ostream& stream) const;
}

std::ostream& operator<<(std::ostream& stream, const Matrix& matrix) {
    matrix.print(stream);
    return stream;
}

Matrix a, b;
std::cout << a << b << "\n";
    \end{lstlisting}
\end{frame}

\begin{frame}[fragile]
    \frametitle{When to overload}
    Three Basic Rules of Operator Overloading in C++
    \begin{itemize}
        \item Only overload if the meaning of the operator is obviously clear and undisputed
        \item Always stick to the operator's well-known semantics
        \item Always provide all out of a set of related operations
    \end{itemize}
    Source: \href{http://stackoverflow.com/q/4421706/3729508}{http://stackoverflow.com/q/4421706/3729508}
\end{frame}

\begin{frame}[fragile]
    \frametitle{How to overload}
    The Decision between Member and Non-member
    \begin{itemize}
        \item Some operators are required to be members
        \item unary operators should be members
        \item if one operand is treated specially (e.g. is modified) should be members
        \item if both operands are treated equally (e.g. addition) should be non-member
        \item if access to private attributes is required should be non-member
        \item try to avoid declaring friend functions
    \end{itemize}
    Source: \href{http://stackoverflow.com/q/4421706/3729508}{http://stackoverflow.com/q/4421706/3729508}
\end{frame}

\begin{frame}[fragile]
    \frametitle{Which Comparison Operators to use}
    \begin{itemize}
        \item standard library algorithms (e.g. \lstinline{std::sort}) will always expect \lstinline{operator<}
        \item users expect other operators if \lstinline{operator<} exists
    \end{itemize}
    \footnotesize
    \begin{lstlisting}[numbers=none]
bool operator==(const X& l, const X& r){ /* actually compare */ }
bool operator!=(const X& l, const X& r){return !operator==(l,r);}
bool operator< (const X& l, const X& r){ /* actually compare */ }
bool operator> (const X& l, const X& r){return  operator< (r,l);}
bool operator<=(const X& l, const X& r){return !operator> (l,r);}
bool operator>=(const X& l, const X& r){return !operator< (l,r);}
    \end{lstlisting}
    Source: \href{http://stackoverflow.com/q/4421706/3729508}{http://stackoverflow.com/q/4421706/3729508}
\end{frame}

\begin{frame}[fragile]
    \frametitle{Conversion Operators}
    \begin{itemize}
        \item conversion to other types (e.g. bool, string) can be overloaded
        \item difference between implicit and explicit conversion
    \end{itemize}
    \footnotesize
    \begin{lstlisting}[numbers=none]
class MyString {
public:
    operator const char*() const; {return data_;} // implicit
    explicit operator const char*() const {return data_;}
private:
    const char* data_;
}
    \end{lstlisting}
    Source: \href{http://stackoverflow.com/q/4421706/3729508}{http://stackoverflow.com/q/4421706/3729508}
\end{frame}

\end{document}
