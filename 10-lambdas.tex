\documentclass[10pt, compress, usenames, dvipsnames]{beamer}

\usetheme{m}
\usepackage{booktabs}
\usepackage[scale=2]{ccicons}
\usepackage{minted}
\usepackage{listings}
\usepackage{scrextend}

% set the default code style
\lstset{
    frame=tb, % draw a frame at the top and bottom of the code block
    tabsize=4, % tab space width
    showstringspaces=false, % don't mark spaces in strings
    numbers=left, % display line numbers on the left
    commentstyle=\color{OliveGreen}, % comment color
    keywordstyle=\color{MidnightBlue}, % keyword color
    stringstyle=\color{WildStrawberry}, % string color
	language=C++,
	basicstyle=\ttfamily
}

\usemintedstyle{trac}

\usepackage{colortbl}
\usepackage{multirow}

\begin{document}

\title{Lambdas}
\subtitle{Things we stole from functional programming}
\date{\today}
\author{Florian Warg, Max Staff}

\maketitle

\begin{frame}[fragile]
    \frametitle{Functors}
    \begin{itemize}
        \item in C++ we can have callable objects
        \item to do this we have to overload the function call operator
        \item any instantiated objects of those classes are called functors
    \end{itemize}
    \begin{lstlisting}[numbers=none]
class AddOne {
public:
    int operator()(int i) { return i + 1; }
};
/* ... */
AddOne myFunctor;
std::cout << myFunctor(41);
    \end{lstlisting}
\end{frame}

\begin{frame}[fragile]
    \frametitle{Closures}
    \begin{itemize}
        \item closures are a concept in functional programming
        \item a closure is a function that can capture variables of its outer scope
        \item closures have implicit identity, state and behavior
    \end{itemize}
\end{frame}

\begin{frame}[fragile]
    \frametitle{Creating closures with functors}
    \begin{itemize}
        \item we can simulate closures with functors
        \item captured variables are stored in private members
    \end{itemize}
    \begin{lstlisting}[numbers=none]
class Closure {
private:
    int val;
    int& ref;
public:
    Closure(int val, int& ref) 
        : val(val), ref(ref) {}
    int operator()() { return val + ref; }
};
/* ... */
int x = 42;
int y = 31;
Closure f(x, y);
std::cout << f();
    \end{lstlisting}
\end{frame}

\begin{frame}[fragile]
    \frametitle{Anonymous functions}
    \begin{itemize}
        \item anonymous functions originate from the lambda calculus
        \item in lambda calculus every function is anonymous
        \item e.g. $ \lambda x. \lambda y. \lambda z. x y z $ is an anonymous function with 3 arguments
        \item anonymous functions are mostly used as arguments for higher order functions
        \item they can also be used to return a function as a result
        \item in C++ anonymous functions that represent closures are called \textbf{lambdas}
    \end{itemize}
\end{frame}

\begin{frame}[fragile]
    \frametitle{Lambdas}
    \begin{lstlisting}[numbers=none]
[captureList](argumentList) {functionBody}
    \end{lstlisting}
    \begin{lstlisting}[numbers=none]
auto plus1 = [](int x) { return x + 1; };
    \end{lstlisting}
    \begin{itemize}
        \item lambda terms create functors of an anonymous type
        \item only \lstinline{auto} can deduce the type
        \item the capture list can be used to capture variables of the enclosing scope by value or reference
    \end{itemize}
    \begin{lstlisting}[numbers=none]
int x = 42;
int y = 1;
// capture implicitly by value
auto f = [=]() { return x + y; };
// capture implicitly by reference
auto g = [&]() { return x + y; }; 
// capture explicitly int x and int& y
auto h = [x, &y] { return x + y; }
    \end{lstlisting}
\end{frame}
\begin{frame}[fragile]
    \frametitle{Higher-order functions}
    \begin{itemize}
        \item higher-order functions are functions that can take functions as arguments or return functions
        \item in standard C this is done via function pointers
        \item C++ uses templates or type erasure instead
    \end{itemize}
    \begin{lstlisting}[numbers=none]
struct Entry {
    int id;
    char name[32];
};
/* ... */
vector<Entry> v = {{3, "Charlie"}, 
                   {1, "Alice"},
                   {2, "Bob"}};
sort(begin(v), end(v), [](Entry& x, Entry& y) {
    return x.id < y.id; });
// v == {{1, "Alice"}, {2, "Bob"}, {3, "Charlie"}};
    \end{lstlisting}
\end{frame}

\begin{frame}[fragile]
    \frametitle{Partial Function Application}
    \begin{itemize}
        \item we can bind parameters of a function and get a new function of smaller arity as a result
        \item lambda expressions are naturally used for that
    \end{itemize}
    \begin{lstlisting}[numbers=none]
int f(int x, int y, int z) {
    return x * x + 2 * y + z;
}
/* ... */
auto partial = [](int z) {
    return f(1, 9, z);
};
    \end{lstlisting}
\end{frame}

\begin{frame}[fragile]
    \frametitle{Constructing higher-order functions}
    \begin{itemize}
        \item functions can return lambdas because they are objects
        \item lambdas can be used as function parameters by using templates
    \end{itemize}
    \begin{lstlisting}[numbers=none]
auto makeDice(int sides) {
    return [=]() {
        random_device r;
        default_random_engine e(r());
        uniform_int_distribution<int> d(1, sides);
        return d(e);
    };
}
/* ... */
auto cube = makeDice(6);
auto octahedron = makeDice(8);
cout << cube() << " " << octahedron() << "\n";
    \end{lstlisting}
\end{frame}

\begin{frame}[fragile]
    \frametitle{Taking lambdas as arguments}
    \begin{itemize}
        \item template argument deduction can find the type of any lambda
    \end{itemize}
    \begin{lstlisting}[numbers=none]
template<class List, class Functor>
void forEach(List l, Functor f) {
    for (auto i : l) {
        f(i);
    }
}
/* ... */
vector<int> v = {1, 2, 3, 4};
auto print = [](int i) { cout << i << ' '; };
forEach(v, print);
    \end{lstlisting}
\end{frame}
\begin{frame}[fragile]
    \frametitle{Generic lambdas}
    \begin{itemize}
        \item we can the keyword \lstinline{auto} as a parameter type to create generic lambdas
        \item generic lambdas accept any type of arguments (as long as they are well-formed)
    \end{itemize}
    \begin{lstlisting}[numbers=none]
auto f = [](auto container) {
    for (auto i : container) {
        cout << i << " ";
    }
}
/* ... */
int a[] = {1, 2, 3, 4};
vector<string> v = {"Lambdas", "are", "great"};
f(a);
f(v);
    \end{lstlisting}

\end{frame}
\end{document}
