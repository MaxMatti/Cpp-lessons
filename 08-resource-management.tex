\documentclass[10pt, compress, usenames, dvipsnames]{beamer}

\usetheme{m}
\usepackage{booktabs}
\usepackage[scale=2]{ccicons}
\usepackage{minted}
\usepackage{listings}
\usepackage{scrextend}

% set the default code style
\lstset{
    frame=tb, % draw a frame at the top and bottom of the code block
    tabsize=4, % tab space width
    showstringspaces=false, % don't mark spaces in strings
    numbers=left, % display line numbers on the left
    commentstyle=\color{OliveGreen}, % comment color
    keywordstyle=\color{MidnightBlue}, % keyword color
    stringstyle=\color{WildStrawberry}, % string color
	language=C++,
	basicstyle=\ttfamily
}

\usemintedstyle{trac}

\usepackage{colortbl}

\begin{document}

\title{Resource Management}
\subtitle{RAII \& Rule of Five}
\date{\today}
\author{Florian Warg, Max Staff}

\maketitle

\begin{frame}[fragile]
    \frametitle{What are resources?}
    \begin{itemize}
    \item a resource can be anything that exists in limited supply
    \item allocated heap memory
    \item thread, locked mutex, ...
    \item open socket, open file, ...
    \item disk space, database connection, ...
    \end{itemize}
\end{frame}

\begin{frame}[fragile]
    \frametitle{Working with system resources}
    \begin{itemize}
        \item system resources are usually obtained via low level C APIs
        \item management involves 2 functions
        \begin{enumerate}
            \item \lstinline{acquire()};
            \item \lstinline{release()};
        \end{enumerate}
    \end{itemize}
    \begin{lstlisting}[numbers=none]
char msg[] = "100: [error] bugs are everywhere!\n";
/* acquire() */
int fd = open("/tmp/log.txt", O_WRONLY | O_APPEND);
/* use resource */
write(fd, msg, sizeof(msg));
/* release() */
close(fd);
    \end{lstlisting}
\end{frame}


\begin{frame}[fragile]
    \frametitle{Leaking resources}
    \begin{itemize}
        \item it is easy to forget about releasing resources
        \item handle to acquired resource may be lost
        \item -> resource leak
    \end{itemize}
    \begin{lstlisting}[numbers=none]
{
    int arrSize;
    cin >> arrSize;
    int* arr = new int[arrSize];
    for (int i = 0; i < arrSize; ++i) {
        arr[i] = i;
    }
}
/* resource leaked here */
    \end{lstlisting}
\end{frame}


\begin{frame}[fragile]
    \frametitle{Leaking resources II: Deadlock}
    \begin{lstlisting}[numbers=none]
int sum = 0;
mutex m;
void f() { m.lock(); sum += 2; m.unlock(); }
void g() { m.lock(); sum += 10; }
thread t1(f); thread t2(g); 
t1.join(); t2.join();
    \end{lstlisting}
    \begin{itemize}
    \item if t2 starts execution before t1, it will cause a deadlock
    \item even worse: t2 might only \textbf{sometimes} start before t1
    \item very hard to debug because it is not always reproducable
    \end{itemize}
\end{frame}

\begin{frame}[fragile]
    \frametitle{RAII}
    \begin{itemize}
    \item \textbf{Resource Acquisition Is Initialization}
    \item bind the life cycle of resources to the life time of an object
    \item resources are acquired during object construction
    \item resources are released during object destruction
    \item -> resource becomes an invariant of the object
    \end{itemize}
\end{frame}

\begin{frame}[fragile]
    \frametitle{RAII with file example}
    \begin{lstlisting}[numbers=none]
class File {
private:
    int fd;
public:
    File(string path) {
        fd = open(path.c_str(), O_RDONLY);
    }
    ~File() { close(fd); }
    ssize_t write(void* buf, size_t size) { ... }
    ssize_t read(void* buf, size_t size) { ... }
};
/* ... */
{
    File f("/tmp/log.txt"); // resource acquired
    char msg[] = "101: [success] Way better!\n";
    f.write(msg, sizeof(msg));
} // resource released
    \end{lstlisting}
\end{frame}

\begin{frame}[fragile]
    \frametitle{RAII with mutex example}
    \begin{lstlisting}[numbers=none]
class LockGuard {
private:
    mutex& mtx;
public:
    LockGuard(mutex& mtx) : mtx(mtx) { mtx.lock(); }
    ~LockGuard() { mtx.unlock(); }
};
mutex m;
int sum = 0;
void f() { LockGuard guard(m); sum += 2; }
void g() { LockGuard guard(m); sum += 10; }
int main() {
    thread t1(f); thread t2(g);
    t1.join(); t2.join();
}
    \end{lstlisting}
\end{frame}
\begin{frame}[fragile]
    \frametitle{RAII with heap storage example}
    \begin{lstlisting}
class HeapArray {
private:
    int* arr;
    size_t size;
public:
    HeapArray(size_t size)
        : arr(new int[size]), size(size) {}
    ~HeapArray() { delete[] arr; }
    int& operator[](size_t i) { return arr[i]; }
};
/* ... */
{
    HeapArray a(1000); // resource acquired
    a[512] = 42;
} // resource released
    \end{lstlisting}
\end{frame}

\begin{frame}[fragile]
    \frametitle{Problematic behavior}
    \begin{itemize}
        \item resource management introduces a new problem
        \item the user does not know that resources are managed
        \item the user expects every class to behave like a fundamental data type
        \item reference semantics should only apply to pointers and references
    \end{itemize}
    \begin{lstlisting}[numbers=none]
HeapArray a(1000);
a[512] = 42;
HeapArray b = a; // user wants to make a copy
b[512] = 0; // user wants to edit 2nd array
cout << "a[512] = " << a[512] << "\n"; // ???
    \end{lstlisting}
\end{frame}

\begin{frame}[fragile]
    \frametitle{Even more problematic behavior}
    \begin{itemize}
        \item when making copies, resources are not an invariant anymore
        \item this \textbf{will} introduce bugs
    \end{itemize}
    \begin{lstlisting}[numbers=none]
void f(HeapArray a) { cout << a[0] << "\n"; }
int main() {
    HeapArray x(10);
    f(x);
    x[3] = 42; // ???
}
    \end{lstlisting}
\end{frame}

\begin{frame}[fragile]
    \frametitle{Rule of Five}
    A class that manages its own resources, must provide 5 special functions in order to implement the semantics of fundamental types
    \begin{enumerate}
        \item destructor
        \item copy constructor
        \item copy assignment operator
        \item move constructor
        \item move assignment operator
        \item (functions to make type swappable)
    \end{enumerate}
\end{frame}

\begin{frame}[fragile]
    \frametitle{Destructor}
    \begin{itemize}
        \item release resources in reverse order of construction
    \end{itemize}
    \begin{lstlisting}[numbers=none]
/* ctor */
HeapArray(size_t size)
    : arr(new int[size]), size(size) {}
/* dtor */
~HeapArray() { delete[] arr; }
    \end{lstlisting}
\end{frame}

\begin{frame}[fragile]
    \frametitle{Copy constructor}
    \begin{itemize}
        \item construct a new object with the same value
    \end{itemize}
    \begin{lstlisting}[numbers=none]
HeapArray(const HeapArray& rhs)
    : arr(new int[rhs.size]), size(rhs.size) {
    copy(rhs.arr, rhs.arr + size, arr);
}
    \end{lstlisting}
\end{frame}

\begin{frame}[fragile]
    \frametitle{Copy assignment operator}
    \begin{itemize}
        \item assign value of other object to this object
        \item usually implemented in terms of copy ctor
    \end{itemize}
    \begin{lstlisting}[numbers=none]
HeapArray& operator=(HeapArray rhs) {
    swap(*this, rhs); // no throw
    return *this;
}
    \end{lstlisting}
\end{frame}

\begin{frame}[fragile]
    \frametitle{C++ concept: Swappable}
    \begin{itemize}
        \item any lvalue or rvalue of \lstinline{T} can be swapped with an lvalue or rvalue of some other type
        \item swapping is done by calling unqualified \lstinline{swap()} in context where \lstinline{std::swap} and user-defined \lstinline{swap} are visible
    \end{itemize}
    \begin{lstlisting}[numbers=none]
struct MyT {
    int x;
    friend void swap(MyT& lhs, MyT& rhs);
};
void swap(MyT& lhs, MyT& rhs) {
    using std::swap;
    swap(lhs.x, rhs.x);
}
    \end{lstlisting}
\end{frame}

\begin{frame}[fragile]
    \frametitle{Move constructor}
    \begin{itemize}
        \item construct new object by moving state from other object
        \item "hand over pointer instead of copying big objects"
        \item other object is left empty (but in valid state)
    \end{itemize}
    \begin{lstlisting}[numbers=none]
HeapArray(HeapArray&& rhs) 
    : arr(move(rhs.arr)), size(move(rhs.size)) {
    rhs.arr = nullptr;
}
    \end{lstlisting}
\end{frame}

\begin{frame}[fragile]
    \frametitle{Move assignment operator}
    \begin{itemize}
        \item move internal state of object to existing object
        \item "hand over pointer instead of copying big objects"
        \item other object is left empty (but in valid state)
    \end{itemize}
    \begin{lstlisting}[numbers=none]
HeapArray& HeapArray::operator=(HeapArray&& rhs) {
    arr = move(rhs.arr);
    size = move(rhs.size);
    rhs.arr = nullptr;
}
    \end{lstlisting}
\end{frame}

\begin{frame}[fragile]
    \frametitle{Rule of Zero}
    \begin{itemize}
        \item assume you compose a new data type
        \item if all of the member types conform to RAII and no further resources are required, the new type conforms to RAII as well
        \item -> in high level C++ none of the RO5 functions have to be implemented
        \item the compiler-generated implementation is sufficient
        \item this is called the \textbf{Rule of Zero}
    \end{itemize}
    \begin{lstlisting}[numbers=none]
struct MyType {
    string str;
    vector<int> vec;
}; // string and vector conform to RAII
MyType x;
MyType y = x; // implicit copy ctor
MyType z = std::move(z); // implicit move assign
    \end{lstlisting}
\end{frame}

\end{document}
