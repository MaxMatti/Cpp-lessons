\documentclass[10pt, compress, usenames, dvipsnames]{beamer}

\usetheme{m}
\usepackage{booktabs}
\usepackage[scale=2]{ccicons}
\usepackage{minted}
\usepackage{listings}
\usepackage{scrextend}

% set the default code style
\lstset{
    frame=tb, % draw a frame at the top and bottom of the code block
    tabsize=4, % tab space width
    showstringspaces=false, % don't mark spaces in strings
    numbers=left, % display line numbers on the left
    commentstyle=\color{OliveGreen}, % comment color
    keywordstyle=\color{MidnightBlue}, % keyword color
    stringstyle=\color{WildStrawberry}, % string color
	language=C++,
	basicstyle=\ttfamily
}

\usemintedstyle{trac}


\begin{document}

\title{cv-qualified types}
\subtitle{Good interface design}
\date{\today}
\author{Florian Warg, Max Staff}

\maketitle

\begin{frame}[fragile]
    \frametitle{Motivation 1}
    \begin{itemize}
        \begin{lstlisting}[numbers=none]
int min(int* arr, size_t len);
        \end{lstlisting}
    \item think of a function that takes an array by pointer
    \item since it is called by reference, the function might alter the array
    \item how can we guarantee the user that we do not alter data?
    \end{itemize}
\end{frame}

\begin{frame}[fragile]
    \frametitle{Motivation 2: Copy operations}
    \begin{itemize}
        \begin{lstlisting}[numbers=none]
class Person {
private:
    string name;
    int age;
public:
    /* ... */
    string getName() { return name; }
};
        \end{lstlisting}
    \item getters (accessors) should allow to read values of private members
    \item imagine "name" to be a huge object (really long string)
    \item every time we check the value we would have to copy and return the whole string
    \item how can we reduce the overhead?
    \end{itemize}
\end{frame}

\begin{frame}[fragile]
    \frametitle{Motivation 2: Reducing overhead?}
    \begin{itemize}
        \begin{lstlisting}[numbers=none]
class Person {
private:
    string name;
    int age;
public:
    /* ... */
    string& getName() { return name; }
};
        \end{lstlisting}
    \item we choose to return the private variable by reference
    \item overhead is reduced from copying the whole object to copying a constant size reference
    \item have we introduced a new problem?
    \end{itemize}
\end{frame}

\begin{frame}[fragile]
    \frametitle{Motivation 2: Violation of OOP rules}
    \begin{itemize}
        \begin{lstlisting}[numbers=none]
Person p("Alice");
cout << p.getName(); // print "Alice"
p.getName() = "";
cout << p.getName(); // print ""
        \end{lstlisting}
    \item why do we use classes instead of structs?
    \item we use getters and setters so the object can validate changes to private members
    \item this example behaves like a struct with public members
    \item can we reduce overhead and still conform to oop rules?
    \end{itemize}
\end{frame}

\begin{frame}[fragile]
    \frametitle{The const qualifier}
    \begin{itemize}
    \item we can mark variables as constant (const)
    \item const variables can not be modified
    \item trying to modify const vars will cause a compiler error
        \begin{lstlisting}[numbers=none]
const T name; // name is const obj of type T
T const name; // name is const obj of type T

const T* name;  // ptr to const T
T const* name;  // ptr to const T

T* const name;  // const ptr to T

const T* const name;  // const ptr to const T
T const* const name;  // const ptr to const T
        \end{lstlisting}
    \item references have the same rules as pointers
    \end{itemize}
\end{frame}

\begin{frame}[fragile]
    \frametitle{A better interface}
    \begin{itemize}
    \item we can modify example 1 to make strong guarantees about the behavior of the function without exposing implemetation
        \begin{lstlisting}[numbers=none]
/* old */
int min(int* arr, size_t len);
/* better */
int min(const int* arr, size_t len);
        \end{lstlisting}
    \item what is const here? do we need more consts?
    \end{itemize}
\end{frame}

\begin{frame}[fragile]
    \frametitle{Using const in classes}
    \begin{itemize}
    \item we use const in example 2 to prevent modification of private variables through getter methods
        \begin{lstlisting}[numbers=none]
/* old */
string& Person::getName() { return name; }
/* better */
const string& Person::getName() { return name; }
        \end{lstlisting}
    \end{itemize}
\end{frame}

\begin{frame}[fragile]
    \frametitle{Declaring constant values}
    \begin{itemize}
    \item const can also be used to declare constants in code
    \begin{lstlisting}[numbers=none]
/* bad because not type-safe */
#define PI 3.1415926
/* better */
const double PI = 3.1415926;
    \end{lstlisting}
    \item those constants can also be used for array sizes
    \begin{lstlisting}[numbers=none]
size_t n = 32;
int arr[n]; // only works with VLA

const size_t N = 32;
int arr2[N]; // works without VLA
    \end{lstlisting}
    \end{itemize}
\end{frame}

\begin{frame}[fragile]
    \frametitle{Making non-const things const}
    \begin{itemize}
    \item the compiler can implicitly cast objects to const objects
    \item although it does not work the other way around
    \item the compiler can always make things "more const"
    \begin{lstlisting}[numbers=none]
void f(const int n) { /* ... */ }
void g(int n) { /* ... */ }
int x = 42;
const int y = 42;
f(x); // int -> const int
f(y); // const int -> const int
g(x); // ERROR - cannot treat const as non-const
g(y); // const int -> const int 
    \end{lstlisting}
    \item \lstinline{const_cast<T>()} can be used to remove const (please do not use this, yet)
    \end{itemize}
\end{frame}

\begin{frame}[fragile]
    \frametitle{Const in classes - attributes}
    \begin{itemize}
    \item remember, const values cannot be changed
    \item remember, attributes have to be constructed before we enter the ctor
    \begin{lstlisting}[numbers=none]
struct X {
    const int N;
};
/* error because no default ctor */
    \end{lstlisting}
    \end{itemize}
\end{frame}

\begin{frame}[fragile]
    \frametitle{Const in classes - attributes}
    \begin{itemize}
    \item use initializer list to set constants
    \item default values in the class definition are also possible
    \begin{lstlisting}[numbers=none]
struct X {
    const int N;
    X() { N = 42; }
};
/* compiler error */

struct X {
    const int N;
    X() : N(42) {}
};
/* OK */
    \end{lstlisting}
    \end{itemize}
\end{frame}

\begin{frame}[fragile]
    \frametitle{Const in classes - methods}
    \begin{itemize}
    \item we can make methods in classes const, too
    \item const methods can not modify state of the object
    \item const methods can be called on const objects
    \begin{lstlisting}[numbers=none]
class Rectangle {
private:
    /* ... */
    void setWidth(double width);
    double getWidth() const { return width; }
};
Rectangle r1;
const Rectangle r2(2, 4);
r1.setWidth(2); // OK
r2.setWidth(2); // error - r2 is const
r1.getWidth();  // OK
r2.getWidth();  // OK - function is const
    \end{lstlisting}
    \end{itemize}
\end{frame}

\begin{frame}[fragile]
    \frametitle{General rules for const}
    \begin{itemize}
    \item use const as much as possible where it makes sense
    \item use const to define type-safe constant values
    \item getters for primitive types return by value
    \item getters for non-primitive types return by const reference
    \item accessor methods can be constant functions
    \item express read-only intent of functions with const reference arguments
    \end{itemize}
\end{frame}

\begin{frame}[fragile]
    \frametitle{volatile qualifier}
    \begin{itemize}
    \item prevent optimization for variable
    \item usually only used for low-level access (OS) or embedded systems
    \begin{lstlisting}[numbers=none]
volatile int x = 42; // prevent optimization
cout << x;
// cannot optimize to cout << 42
// because x might have been changed 
    \end{lstlisting}
    \end{itemize}
\end{frame}

\begin{frame}[fragile]
    \frametitle{mutable qualifier}
    \begin{itemize}
    \item mutable attributes of a const object can still be changed
    \item it is generally a bad idea to mess with constness
    \item mutable has almost no applications besides debugging
    \begin{lstlisting}[numbers=none]
struct ShopItem {
    int id;
    string name;
    mutable int priceCents;
};

const ShopItem item = { 0, "Club-Mate", 120 };
item.id = 1; // ERROR
item.priceCents = 140; // OK
    \end{lstlisting}
    \end{itemize}
\end{frame}

\end{document}
